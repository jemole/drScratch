\documentclass[12pt]{article}
\usepackage[utf8]{inputenc}
\renewcommand{\familydefault}{\rmdefault}
\usepackage[landscape,a4paper]{geometry}
\geometry{verbose,tmargin=4cm,bmargin=0cm,lmargin=0cm,rmargin=0cm}
\usepackage{graphicx}
\usepackage{eso-pic}

\newcommand\BackgroundPic{%
\put(0,0){%
\parbox[b][\paperheight]{\paperwidth}{%
\vfill
\centering
\includegraphics[width=\paperwidth,height=\paperheight,%
keepaspectratio]{background1.jpg}%
\vfill
}}}


\begin{document}
\AddToShipoutPicture{\BackgroundPic}
~
\vspace{1.2cm}
~
\begin{center}


\LARGE{O equipo do Dr Raspadinha } \\
\Large{é o orgullo de ofrecer esta}\\
\vspace{1cm}
\fontsize{50}{60}{\textbf{CERTIFICADO}}

\vspace{0.4cm}

\Huge{para o proxecto \textbf{%pointname
}}

\vspace{0.4cm}

\Huge{obtiveron unha puntuación de
% \textbf{%pointdni}
}

\vspace{0.3cm}

\Huge{\textbf{%pointcalification
}}

\vspace{0.3cm}

\Large{na análise na ferramenta Dr Scratch (www.dr.scratch.org).}

\vspace{.2cm}

%\Large{Y para que as\'i conste se expide el presente certificado}
%
\vspace{1cm}


\scriptsize
{A ferramenta Dr. Scratch ten por obxectivo proporcionar un medio de aprendizaxe e feedback sobre a calidade dos proxectos en perigo. \\
}

\ \\

\end{center}
\end{document}
